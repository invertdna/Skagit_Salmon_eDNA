\documentclass[]{article}
\usepackage{lmodern}
\usepackage{amssymb,amsmath}
\usepackage{ifxetex,ifluatex}
\usepackage{fixltx2e} % provides \textsubscript
\ifnum 0\ifxetex 1\fi\ifluatex 1\fi=0 % if pdftex
  \usepackage[T1]{fontenc}
  \usepackage[utf8]{inputenc}
\else % if luatex or xelatex
  \ifxetex
    \usepackage{mathspec}
  \else
    \usepackage{fontspec}
  \fi
  \defaultfontfeatures{Ligatures=TeX,Scale=MatchLowercase}
\fi
% use upquote if available, for straight quotes in verbatim environments
\IfFileExists{upquote.sty}{\usepackage{upquote}}{}
% use microtype if available
\IfFileExists{microtype.sty}{%
\usepackage[]{microtype}
\UseMicrotypeSet[protrusion]{basicmath} % disable protrusion for tt fonts
}{}
\PassOptionsToPackage{hyphens}{url} % url is loaded by hyperref
\usepackage[unicode=true]{hyperref}
\hypersetup{
            pdfborder={0 0 0},
            breaklinks=true}
\urlstyle{same}  % don't use monospace font for urls
\usepackage{longtable,booktabs}
% Fix footnotes in tables (requires footnote package)
\IfFileExists{footnote.sty}{\usepackage{footnote}\makesavenoteenv{long table}}{}
\IfFileExists{parskip.sty}{%
\usepackage{parskip}
}{% else
\setlength{\parindent}{0pt}
\setlength{\parskip}{6pt plus 2pt minus 1pt}
}
\setlength{\emergencystretch}{3em}  % prevent overfull lines
\providecommand{\tightlist}{%
  \setlength{\itemsep}{0pt}\setlength{\parskip}{0pt}}
\setcounter{secnumdepth}{0}
% Redefines (sub)paragraphs to behave more like sections
\ifx\paragraph\undefined\else
\let\oldparagraph\paragraph
\renewcommand{\paragraph}[1]{\oldparagraph{#1}\mbox{}}
\fi
\ifx\subparagraph\undefined\else
\let\oldsubparagraph\subparagraph
\renewcommand{\subparagraph}[1]{\oldsubparagraph{#1}\mbox{}}
\fi

% set default figure placement to htbp
\makeatletter
\def\fps@figure{htbp}
\makeatother


\usepackage[margin=1in]{geometry} % make the margins 1 inch on all sides of the document

\date{}

\begin{document}


\textbf{1. Project Title:}{~}

Improving techniques for estimating abundance and habitat use in
nearshore marine habitats using environmental DNA.

\textbf{2. Investigators:{~}}

A. Ole Shelton (NWFSC), Ryan Kelly (University of Washington), Correigh
Greene (NWFSC), Linda Park (NWFSC)

\textbf{3. Project Duration and Requested Funds:{~}}

2 years, 112.5K per year.

\textbf{4. Summary:{~}}

Organisms of all kinds shed cells containing diagnostic DNA into the
environment, which can be recovered and assigned to a taxon based upon
its match to known sequences. Because DNA degrades under most ambient
environmental conditions---the half-life of DNA in fresh- and saltwater
is approximately 24-48 hours (\emph{1,2})---this environmental DNA
(eDNA) provides a snapshot of the species recently present in the
sampled habitat. However, while it is widely accepted that DNA can be
collected and identified from a range of environmental samples,
connecting field collections of eDNA with abundance surveys remains
largely unexplored.

{~}Here, we propose to develop eDNA survey methods to quantify fish
communities (with a focus on salmon, herring, and smelt species) in a
nearshore estuarine habitat in Puget Sound.{~ }To compare the efficacy
of eDNA and traditional methods, we will collect water samples in
parallel with collections made via three traditional net sampling
methods, targeting nearshore fish communities that provide a range of
spatial sampling scales (from meters to 100s of meters). We will use
both quantitative PCR (qPCR) and massively parallel DNA sequencing
technologies to provide eDNA data. We will then apply a newly developed
statistical framework to provide field estimates of the relationship
between species abundance and eDNA. Our replicated sampling
design---using three field methods at three spatial scales---provides an
opportunity to understand the appropriate spatial scale for eDNA
sampling, and the potential value and pitfalls of eDNA surveys for
understanding patterns of fish abundance.

\textbf{}

\textbf{5. Scope, Objectives, Merit:} {~}

{{ }}Early life-history is a critical period for most commercially
important fishes. For anadromous salmonids, the transition from
freshwater to the marine environment is a key determinant of marine
survival and fisheries productivity. Similarly, ecologically important
forage fish (e.g. herring and smelt) use nearshore and estuarine
habitats as spawning and rearing grounds. Thus, accurate estimates of
fish abundance in nearshore estuarine areas are critical to
understanding early life-history survival and play an important, poorly
understood role in driving stock assessment models. However, estimates
of fish abundance in the nearshore habitats are difficult and expensive
to obtain using traditional sampling methods; shallow water and
vegetation interfere with acoustic surveys, turbid water often hinders
visual surveys, and the presence of vegetation and other structures
restricts the efficacy of some net survey techniques. Nonetheless,
estimating salmon and other fish species' abundance is especially
important in light of (a) continued loss of foundational vegetated
habitats such as seagrass beds, salt marshes, and other coastal
wetlands, and (b) restoration efforts intended to mitigate such losses.
Assessing the importance of nearshore habitats to salmonids and forage
fish ---and the success of nearshore restoration efforts in
particular---requires efficient methods for quantifying abundance in
these habitats. \textbf{We propose to apply recently developed
environmental DNA (eDNA) survey techniques to assess the fish
communities across three habitats used at different points during the
salmon migration from freshwater to the ocean.} We will characterize the
abundance of salmonids and other commercially valuable and ecologically
important fish in three distinct estuarine habitats; these three
habitats offer the additional benefit of comparing results across three
different spatial sampling techniques. If successful, this project would
adapt a rapidly-developing, innovative technology that could inform
stock assessments nationwide.

This project has three main objectives: \textbf{(1)} Validate and
improve eDNA methods for rapidly detecting the occurrence and abundance
of ecologically important coastal fish; \textbf{(2)} Assess the efficacy
of eDNA methods by comparing estimates of occurrence and abundance from
traditional net sampling and eDNA methods across three existing sampling
methodologies; \textbf{(3)} Compare costs and relative benefits of eDNA
methodologies relative to traditional sampling methods. The project will
improve abundance estimates for species that are ecologically and
commercially important but difficult to survey, including juvenile
salmon (\emph{Oncorhynchus} spp.) and forage fish (e.g. herring
(\emph{Clupea pallasi}) and smelt (family \emph{Osmeridae})). Herring
and smelt have been estimated as the most numerically abundant fish but
exhibit recent declines, while Chinook salmon populations in this river
system are ESA-listed, but suffer from poor quantification (\emph{3,4}).

{ }We will focus on developing rigorous eDNA methods in a single estuary
and will compare results from eDNA samples with three widely used net
sampling techniques (Fyke net, beach seine, surface trawl). Each of
these sampling methods occurs in distinct habitats and samples different
cumulative areas (increasing from the scale of \textasciitilde{}1 to
100s of meters); as such, we expect the match between eDNA and net
sample to vary with each sampling method. Concordance should be highest
for Fyke nets and lowest for surface trawls. However, the rate at which
the concordance between eDNA and sampling scale changes is itself a
useful metric because it has direct implications for the appropriate
scale at which to apply eDNA methods in the field. Beyond individual
sample-to-sample comparisons, we will calculate aggregative measures of
fish density from traditional and eDNA methods. Such sample aggregation
is a critical step in the development of abundance indices that feed
directly into most stock assessments. Thus our research plan provides
information on two pressing questions for the future use of eDNA in
stock assessments: (1) at what spatial scale can eDNA accurately reflect
local abundance? and (2) can eDNA provide integrated metrics of
abundance on scales useful for management?

Developing quantitative applications, such as we propose here, is the
key next step in the evolution of eDNA into a practically useful tool,
pointing the way to such uses as stock assessments, counts of endangered
or invasive species, and other quantitative surveys for species and
communities of interest. Accordingly, the work we propose here applies
NMFS-wide, and additionally has benefits that will ramify outside of the
agency. For example, USGS and State agencies have expressed interest in
surveying salmonids and protected species with eDNA (pers. comm.), work
that the proposed project would directly inform. \textbf{In short, we
propose to lay the necessary methodological and quantitative groundwork
to make eDNA useful for NMFS and others.} An added benefit of eDNA
methods for NMFS is the potential to bring down the future costs of
survey work: on a per-sample basis, eDNA appears likely to become
cheaper than many traditional sampling methods. Finally, the project
would contribute to a durable collaboration between the NWFSC and UW
researchers in the College of the Environment, leveraging NMFS's
financial and human resources.

\textbf{6. Defined Uncertainties:{~}}

The idea that one can sample a volume of water, sequence the DNA
present, and report what species are living nearby is widely accepted
among microbial biologists (e.g \emph{5,6}). For fisheries ecologists
that have historically use manual count data, eDNA has quickly become a
potential new avenue through which to examine the world, but has yet to
come into common practical use because of unknowns surrounding
quantification.{ Preliminary data in hand demonstrate eDNA's
feasibility, appropriate spatial scale, and suitable taxonomic breadth
for the proposed project.}

To date, no eDNA study has explicitly linked biomass to field estimates
under field conditions and very few have linked them under controlled
laboratory conditions (\emph{7}). Instead, most researchers have either
asserted that the proportion of sequences observed from environmental
samples mirrors the abundance (either count or biomass) of physically
collected individuals (\emph{6}). While these assertions may accurately
reflect a functional link between individuals and DNA in the
environment, a diverse set of processes that separate the biomass of
source animals and the observed DNA fragments means that there are a
large number of ways to arrive at spurious correlations between eDNA and
observed catches.{~}

A pervasive concern in the eDNA literature is determining the
appropriate spatial scale for eDNA studies. Current evidence suggests
that eDNA can distinguish ecological communities at scales of 60-100m,
even in a dynamic marine nearshore environment (Fig. 1), and is useful
for detecting even rare species (\emph{1,8}). Our proposed work advances
eDNA methods by providing a link between fish abundance and eDNA surveys
and an application for rapidly assessing nearshore habitat use by fish.

At present, methods for eDNA are not sufficiently well developed to make
full inference about density or biomass in an ecological community from
eDNA alone. Similar challenges confront estimation of density and
biomass based on traditional sampling methods (\emph{9,10,11}), but do
not prevent researchers from making the best approximations possible
given existing knowledge and data. We will apply a newly developed
Bayesian statistical framework to assess uncertainties in linking
biomass to eDNA reads, leveraging a large body of statistical thought
from the fisheries literature and analogizing eDNA to the use of a new
``net'' used to sample target fish species.{~}

\textbf{7. NMFS wide concern:}{~}

eDNA has widespread applicability for ecosystem-based management in all
NMFS regions because of its potential to assess many species present in
an area, not just the target fishery species. More immediately, methods
development for salmon have application to ESA-listed species in three
regions (Northeast, Northwest, and Southwest). The noninvasive nature of
eDNA sampling should be especially useful for fisheries that have been
curtailed due to overfishing---making existing data on the status of the
target species extremely limited---as well as for species where
gear-avoidance or difficult habitats interfere with traditional
assessment methods. The presence of eDNA is also independent of species
or gear-type of a fishery; thus development of laboratory tools and
analysis methods could be used to augment assessments of a wide range of
species including forage fish, groundfish, and crustaceans; for example,
preliminary eDNA surveys could maximize cost-effectiveness in the
immediate future by highlighting spatial areas in which to focus manual
sampling. One long-term potential application for this method would be
the development of autonomous samplers that could be deployed to collect
eDNA (water) repeatedly over time to provide a detailed temporal picture
of fish abundance and movement. {~}

Our proposal addresses ASTWG themes \textbf{2} (Remote species
identification and enumeration) and \textbf{5} (Efficient Ecosystem
Surveys), and is broadly applicable within NMFS, with potential uses in
stock assessments, counts of endangered or invasive species, and other
quantitative surveys for species and communities of interest.

\textbf{8. Technical Approach:}

Our approach has three components: 1) Field collection of eDNA samples
conducted in parallel with existing nearshore sampling using three
sampling techniques. 2) Laboratory processing of eDNA samples to produce
quantitative estimates of DNA abundance and 3) application of novel
statistical approaches to generate defensible estimates of biomass from
eDNA and comparison of estimates from eDNA to those derived from
traditional sampling methods. We discuss each in turn.

\textbf{\emph{Field collections}.} This project will align with the
Skagit River Intensively Monitored Watershed Project, which tracks
status and trends of species in the Skagit River estuary and Bay (Fig.
2), including all species of Pacific salmon and several forage fish
species (\emph{12}). The focal species is Chinook salmon, which is
listed as threatened under the Endangered Species Act. Small
(\textless{}50 mm) Chinook leave freshwater and rear in the estuary
before migrating to the Pacific Ocean (\emph{3}). Substantial variation
in migration timing and fish movement complicates traditional
estimations of abundance; nevertheless these data represent the longest
time series of juvenile abundance Puget Sound salmon and are vital for
determining stock status, trends, and responses to habitat restoration
efforts.{~}

The Skagit Intensively Monitored Watershed Project counts Chinook salmon
at four life stages. We will focus on three stages: estuary residence in
channels and impoundments, nearshore intertidal residence along beaches
and in lagoons, and subtidal residence in Skagit Bay (Fig. 2, Table 1).
Sampling in each habitat requires different sampling gear and
complicates abundance estimates across habitats. In estuary channels and
nearshore lagoons, capture efficiency can be very high
(\textgreater{}50\%), while efficiency in other habitats is much lower
(\textless{} 10\%). These various sampling procedures highlight the
potential utility of eDNA for estimating local abundance, calibrating
eDNA to each procedure individually.{~}

Beyond salmonids, Pacific herring, surf smelt, stickleback, and Pacific
sandlance reside in Skagit Bay. The multi-species eDNA techniques we
propose will capture these and other species inhabiting estuarine and
nearshore habitats at various times of the year.

We will collect three 1L water (eDNA) samples in each of three seasons
(February-April, May-July, and August-October). The first two seasons
represent prime sampling windows when Chinook salmon and forage fish are
abundant in the estuary and bay; fall sampling will serve as negative
controls. Water samples will be taken in triplicate at each location. We
will record environmental covariates to assess their effects on
concordance between eDNA and net sampling.

\textbf{\emph{Environmental DNA methods.}} There are two distinct
approaches for eDNA analysis. In the first, the amount DNA from a single
target taxon is quantified using quantitative PCR (qPCR; \emph{8}), by
comparing the amplification rate of a field sample to one of a standard
of known concentration. This protocol quantitatively assesses changes in
single target-species' DNA concentrations in the field (\emph{13,14}).

In the second technique a single locus is PCR-amplified from all genomes
present in a sample, the resulting products (amplicons) are sequenced,
and the resulting sequences are matched to those of known species in a
large database (\emph{15}). Amplicon sequencing can provide data for a
hundreds of taxa in the sampled community, but only provides information
about relative abundance of DNA in the sample.{~}

We propose to combine these two methods, using 1) qPCR to quantify the
abundance of key species, and 2) amplicon sequencing to provide the
relative abundance for dozens of species in the community, and then 3)
linking qPCR and sequencing results to provide the first quantitative
survey of an entire community (\emph{16}).

For our focal species (salmonids and forage fish), there is sufficient
published and unpublished genetic information to reliably identify taxa
to the species level with both qPCR and sequencing approaches. \emph{14}
provides qPCR primers for Chinook salmon and colleagues at the US
Geological Survey have developed qPCR primers specific for eight
salmonid species (J. Duda pers. comm.). We will choose the three species
most frequently observed in the previous year's net surveys for qPCR,
and quantify eDNA using replicate qPCRs from each of the triplicate
water samples.{~}

We will then use three sets of primers to generate mixed-species
amplicons for sequencing. We will use primers with the following target
genes and taxa: 16S mtRNA (16S; targeting fish and a diverse range of
invertebrate taxa in Puget Sound; \emph{16}), 12S mtDNA (targeting
vertebrates; \emph{17}), and we will use the software ecoPrimers
(\emph{18}) to develop a novel primer set that will amplify a region of
mitochondrial cytochrome c oxidase I (CO1), which varies consistently
among the three target taxa chosen for qPCR.{~}

\textbf{\emph{Linking eDNA and net surveys.}} Matching \textbf{} eDNA
field sampling and existing net sampling protocols will reveal the
relationship between fish abundance and eDNA. While there are challenges
for translating observations of eDNA into biomass, these are largely
analogous to those faced by traditional sampling methods (Fig. 3;
\emph{16}); we view eDNA sampling as a new ``net'' with a unique set of
traits that can produce statistical biases in the estimation of
abundance. Elsewhere the PIs have developed a statistical framework for
quantifying the stages connecting biomass to observed eDNA counts (Fig.
3, \emph{16}). This proposal provides an ideal application of these new
methods and in a management relevant setting.

An unresolved issue is how DNA is shed and disperses in the environment
(\emph{14}). Our multi-scale field sampling approach allows us to
explore the relationship between eDNA and fish abundance across spatial
scales. We expect the strongest eDNA-net correlations to be in estuarine
channels, and weakest correlations to be in offshore trawls, consistent
with the spatial scales of sampling in those habitats.{~}

Beyond individual eDNA-to-net-sample comparisons, we will also compare
aggregated estimates of abundance. In stock assessments, such abundance
indices provide estimates of overall average density and trends through
time; identical procedures can be performed with eDNA data. Thus while a
single eDNA sample may not perfectly reflect abundance observed in an
adjacent sample, on an aggregate basis both traditional and eDNA methods
should provide equivalent estimates of mean abundance of the fish
community. We expect strong concordance between aggregate estimates
across all three scales of investigation.

\textbf{9. Literature Cited:}

\begin{itemize}
\tightlist
\item
  Thomsen, P. F. \emph{et al.} Monitoring endangered freshwater
  biodiversity using environmental DNA. 2012. \emph{Molecular Ecology}
  21, 2565--2573.
\item
  Turner, C. R. \emph{et al.} 2014. Particle size distribution and
  optimal capture of aqueous macrobial eDNA. \emph{Methods in Ecology
  and Evolution} 5:676--68.
\item
  Beamer, E., A. et al. 2005. Delta and nearshore restoration for the
  recovery of wild Skagit River Chinook salmon: Linking estuary
  restoration to wild Chinook salmon populations. Supplement to Skagit
  Chinook Recovery Plan, Skagit River System Cooperative, LaConner, WA.
  Available at: www.skagitcoop.org.
\item
  Zimmerman, M. S., C. Kinsel, E. Beamer, E. Conner, and D. Pflug. 2015.
  Abundance, survival, and life history strategies of juvenile migrant
  Chinook in the Skagit River, Washington. \emph{Transactions of the
  American Fisheries Society} 144:627--641.
\item
  Venter, J.C., \emph{et al.} 2004. Environmental genome shotgun
  sequencing of the Sargasso Sea. \emph{Science} 304:66-74.
\item
  deVargas, C., et al. 2015. Eukaryotic plankton diversity in the sunlit
  ocean. \emph{Science} \emph{348:} 1261605.
\item
  Takahara, T., T. Minamoto, H. Yamanaka, H. Doi, Z. Kawabata. 2012.
  Estimation of fish biomass using environmental DNA. \emph{PLoS ONE},
  \emph{7:}e35868.
\item
  Jerde, C.L., A.R. Mahon, W.L. Chadderton, D.M Lodge. 2011.
  `Sight-unseen' detection of rare aquatic species using environmental
  DNA. \emph{Conservation Letters} 4:150--157.
\item
  Burnham, K.P. D.R. Anderson, J.L. Laake. 1980. Estimation of density
  from line transect sampling of biological populations. \emph{Wildlife
  Monographs} 72:3-202.
\item
  Hankin, D.G., G.H. Reeves 1988. Estimating total fish abundance and
  total habitat area in small streams basesd on visual estimation
  methods. \emph{Canadian Journal of Fisheries and Aquatic Science}
  45:834-844.
\item
  Kéry, M., J.A. Royle. 2010. Hierarchical modelling and estimation of
  abundance and population trends in metapopulation designs.
  \emph{Journal of Animal Ecology} 79:453--461.
\item
  Greene, C.M., L. Kuehne, C. Rice, K. Fresh, D. Penttila. 2015. Forty
  years of change in forage fish and jellyfish abundance across greater
  Puget Sound, Washington (USA): anthropogenic and climate associations.
  \emph{Marine Ecology Progress Series} 525:153-170.
\item
  Goldberg, C.S., D.S. Pilliod, R.S. Arkle, L.P. Waits. 2011 Molecular
  detection of vertebrates in stream water: a demonstration using Rocky
  Mountain tailed frogs and Idaho giant salamanders. \emph{PloS one}
  6:e22746.
\item
  Laramie, M.B., D.S. Pilliod, C.S. Goldberg. 2015. Characterizing the
  distribution of an endangered salmonid using environmental DNA
  analysis. \emph{Biological Conservation} 183:29-37.
\item
  Leray, M., N. Knowlton. 2015. DNA barcoding and metabarcoding of
  standardized samples reveal patterns of marine benthic diversity.
  Proc. Natl. Acad. Sci. {~112:2076--2081}.
\item
  Shelton, A.O., J.L. O'Donnell, J.F. Samhouri, N. Lowell, G. Williams,
  R.P. Kelly. \emph{In Revision.} A statistical framework for inferring
  biological communities from environmental DNA\emph{. Ecological
  Applications}.
\item
  Port, J. A. \emph{et al. In Review.} Assessing the vertebrate
  community of a kelp forest ecosystem using environmental DNA.
  Molecular Ecology.
\item
  Riaz, T. \emph{et al.} 2011. ecoPrimers: inference of new DNA barcode
  markers from whole genome sequence analysis. \emph{Nucleic Acids
  Research} 39:e145--e145.
\end{itemize}

\textbf{10. Expected Results:{~}}

We expect to produce a quantitative assessment of an advanced sampling
technique that, if properly validated, could significantly extent NMFS's
capacity to survey target species and to conduct ecosystem assessments.
We will lay necessary groundwork to make eDNA useful for NMFS and
others. We believe this is reasonably accomplishable based upon existing
and ongoing work in our laboratory group, which has included sequencing
over 300 individual PCR products from Puget Sound water samples; our
team brings significant field-sampling and eDNA experience to bear on
the question of how traditional and genetic sampling methods can be
cross-validated.

\begin{itemize}
\tightlist
\item
  {Year 1 Deliverables:} Broad-scale report on fish communities from
  eDNA, validated proof-of-concept, and molecular tools for future
  use.{~ }Preliminary comparison of eDNA and net sampling for all three
  methods.
\item
  {Year 2 Deliverables}: Full description of traditional and eDNA
  samples for fish community. Comparisons of aggregate abundance metrics
  and indices of abundance. Cost-benefit comparison of eDNA and net
  methodologies.
\end{itemize}

\textbf{11. Probability of success} (200 word limit).{~}

Year 1 deliverables have a high probability of success, given
preliminary data (Fig. 1) and recent experience with the relevant
techniques. eDNA validation with existing methods is straightforward
using the statistical framework developed by the PIs in (\emph{16}).
Preliminary data from Puget Sound include sequences from many of the
target taxa and other interacting species in the nearshore community.

Year 2 deliverables are less certain, given unknown performance of
additional molecular tools; mapping the spatial and temporal dynamics of
key taxa is very likely to succeed. However, the simultaneous use of
multiple species-specific markers across multiple spatial scales will
inform appropriate spatial and temporal scales of investigation for
eDNA, and the work will reflect these scales. We will endeavor to reduce
these key uncertainties with rigorous \emph{in silico} and lab-based
testing (for new markers) and by leveraging existing research with
partners working in Monterey and Puget Sound to validate markers in
development. Despite these uncertainties, marrying molecular and
traditional surveys holds great promise for improving ecosystem surveys
and stock assessments alike. Cost-benefit analysis is straightforward,
with high probability of success.\textbf{}

\textbf{12. Schedule of Project Milestones:{~}}

\begin{longtable}[]{@{}lll@{}}
\toprule
\begin{minipage}[t]{0.30\columnwidth}\raggedright\strut
\textbf{Year}\strut
\end{minipage} & \begin{minipage}[t]{0.30\columnwidth}\raggedright\strut
\textbf{Date}\strut
\end{minipage} & \begin{minipage}[t]{0.30\columnwidth}\raggedright\strut
\textbf{Activities}\strut
\end{minipage}\tabularnewline
\begin{minipage}[t]{0.32\columnwidth}\raggedright\strut
2016\strut
\end{minipage} & \begin{minipage}[t]{0.32\columnwidth}\raggedright\strut
Spring-Summer{~}\strut
\end{minipage} & \begin{minipage}[t]{0.32\columnwidth}\raggedright\strut
Receive grant{~}

Hire postdoctoral researcher

Collect field water samples in conjunction with Skagit Intensively
Monitored Watershed Project\strut
\end{minipage}\tabularnewline
\begin{minipage}[t]{0.32\columnwidth}\raggedright\strut
\strut
\end{minipage} & \begin{minipage}[t]{0.32\columnwidth}\raggedright\strut
Fall{~}\strut
\end{minipage} & \begin{minipage}[t]{0.32\columnwidth}\raggedright\strut
Complete 2016 eDNA collections.

Initiate laboratory work.

{~ ~ ~ }(qPCR, marker development, and sequencing)\strut
\end{minipage}\tabularnewline
\begin{minipage}[t]{0.32\columnwidth}\raggedright\strut
2017\strut
\end{minipage} & \begin{minipage}[t]{0.32\columnwidth}\raggedright\strut
Winter{~}\strut
\end{minipage} & \begin{minipage}[t]{0.32\columnwidth}\raggedright\strut
Perform laboratory assays of qPCR for 2016{~}

Complete processing of sequencing samples, send to sequencing center,
perform primary bioinformatics analyses

eDNA collections (Feb. 2017); bioinformatics\strut
\end{minipage}\tabularnewline
\begin{minipage}[t]{0.32\columnwidth}\raggedright\strut
\strut
\end{minipage} & \begin{minipage}[t]{0.32\columnwidth}\raggedright\strut
Spring{~}\strut
\end{minipage} & \begin{minipage}[t]{0.32\columnwidth}\raggedright\strut
eDNA collections

Perform preliminary comparison of eDNA and net sampling results.\strut
\end{minipage}\tabularnewline
\begin{minipage}[t]{0.32\columnwidth}\raggedright\strut
\strut
\end{minipage} & \begin{minipage}[t]{0.32\columnwidth}\raggedright\strut
Summer{~}\strut
\end{minipage} & \begin{minipage}[t]{0.32\columnwidth}\raggedright\strut
Final eDNA collections and lab processing; sequencing\strut
\end{minipage}\tabularnewline
\begin{minipage}[t]{0.30\columnwidth}\raggedright\strut
\strut
\end{minipage} & \begin{minipage}[t]{0.30\columnwidth}\raggedright\strut
Fall{~}\strut
\end{minipage} & \begin{minipage}[t]{0.30\columnwidth}\raggedright\strut
Final statistical and bioinformatics analysis{~}\strut
\end{minipage}\tabularnewline
\begin{minipage}[t]{0.30\columnwidth}\raggedright\strut
2018{ }\strut
\end{minipage} & \begin{minipage}[t]{0.30\columnwidth}\raggedright\strut
Winter 2018\strut
\end{minipage} & \begin{minipage}[t]{0.30\columnwidth}\raggedright\strut
Complete analysis and write results for publication\strut
\end{minipage}\tabularnewline
\bottomrule
\end{longtable}

\textbf{13. Expertise of Principle Investigators and Partnerships:} (500
word limit).{~}

Shelton will be in charge of overseeing the statistical analysis of eDNA
data and the development of new methods for application as necessary.
Shelton has experience developing and applying new statistical methods
to environmental and population data. Shelton will work with Greene to
develop appropriate metrics for linking and comparing traditional and
eDNA data.{~}

Greene will oversee field sampling, consistent with his role with the
Skagit Intensively Monitored Watershed Project and will use his
statistical expertise to aid the postdoc and Shelton in analyses.{~}

Park and Kelly will oversee the molecular aspects of the study including
proper sample collection, preparation, molecular techniques, and
bioinformatics analysis. Park has extensive expertise developing and
analyzing environmental and tissue-derived genetic samples for use in
practical NMFS applications. Kelly has experience developing and using
eDNA amplicon-sequencing protocols for marine ecosystem surveys over the
past several years in Monterey, CA, and Puget Sound, WA.

Postdoc, TBD, will carry out day-to-day project activities, and will be
responsible for coordinating activities among the principal
investigators. Relevant expertise will include molecular and
bioinformatics techniques, and familiarity with field methods and taxa.

\textbf{}

\textbf{14. Investigators and Affiliations:}

\begin{longtable}[]{@{}ll@{}}
\toprule
\begin{minipage}[t]{0.47\columnwidth}\raggedright\strut
\textbf{Investigator}\strut
\end{minipage} & \begin{minipage}[t]{0.47\columnwidth}\raggedright\strut
\textbf{Affiliation}\strut
\end{minipage}\tabularnewline
\begin{minipage}[t]{0.47\columnwidth}\raggedright\strut
A. Ole Shelton\strut
\end{minipage} & \begin{minipage}[t]{0.47\columnwidth}\raggedright\strut
NWFSC, Conservation Biology Division\strut
\end{minipage}\tabularnewline
\begin{minipage}[t]{0.48\columnwidth}\raggedright\strut
Ryan P. Kelly\strut
\end{minipage} & \begin{minipage}[t]{0.48\columnwidth}\raggedright\strut
University of Washington,{~}

School of Marine and Environmental Affairs\strut
\end{minipage}\tabularnewline
\begin{minipage}[t]{0.47\columnwidth}\raggedright\strut
Correigh Greene\strut
\end{minipage} & \begin{minipage}[t]{0.47\columnwidth}\raggedright\strut
NWFSC, Fish Ecology Division\strut
\end{minipage}\tabularnewline
\begin{minipage}[t]{0.47\columnwidth}\raggedright\strut
Linda Park\strut
\end{minipage} & \begin{minipage}[t]{0.47\columnwidth}\raggedright\strut
NWFSC, Conservation Biology Division\strut
\end{minipage}\tabularnewline
\bottomrule
\end{longtable}

\textbf{15. Cost Estimates:} (in table format).{~}

\begin{longtable}[]{@{}ll@{}}
\toprule
\begin{minipage}[t]{0.48\columnwidth}\raggedright\strut
\textbf{Itemized Annual Budget}\strut
\end{minipage} & \begin{minipage}[t]{0.48\columnwidth}\raggedright\strut
\textbf{}\\
\strut
\end{minipage}\tabularnewline
\begin{minipage}[t]{0.47\columnwidth}\raggedright\strut
Contract postdoc for laboratory and field work (incl. indirect)\strut
\end{minipage} & \begin{minipage}[t]{0.47\columnwidth}\raggedright\strut
92,000\strut
\end{minipage}\tabularnewline
\begin{minipage}[t]{0.47\columnwidth}\raggedright\strut
Molecular supplies: DNA extraction supplies, primers for multiplexing,
DNA library preparation and QA/QC, sampling bottles, filters, etc.\strut
\end{minipage} & \begin{minipage}[t]{0.47\columnwidth}\raggedright\strut
10,500\strut
\end{minipage}\tabularnewline
\begin{minipage}[t]{0.47\columnwidth}\raggedright\strut
Contract for sequencing service and associated supplies\strut
\end{minipage} & \begin{minipage}[t]{0.47\columnwidth}\raggedright\strut
10,000\strut
\end{minipage}\tabularnewline
\begin{minipage}[t]{0.47\columnwidth}\raggedright\strut
\strut
\end{minipage} & \begin{minipage}[t]{0.47\columnwidth}\raggedright\strut
\textbf{{ }\$112,500}\strut
\end{minipage}\tabularnewline
\bottomrule
\end{longtable}

\textbf{}\\

\textbf{16. Budget Justification:} {~}

The proposed research will require personnel dedicated full-time to this
project, thus the majority of the budget supports a postdoctoral
researcher who will be performing field collections, laboratory
procedures, and data analyses. Supply costs for field and laboratory
work are included, with the largest non-personnel expense being the
contract for next-generation sequencing services. While this expense
might appear large the power of the technique lies in the tens of
millions of sequences that will be generated and the actual cost per
sample (and per sequence) is quite low. {~}

\textbf{}

\textbf{17. Supporting Figures and Tables:{~}}

\textbf{}\\

\textbf{Table 1}. Sampling efforts focused on Chinook salmon in the
Skagit River Intensively Monitored Watershed Project.

\begin{longtable}[]{@{}lllllll@{}}
\toprule
\begin{minipage}[t]{0.12\columnwidth}\raggedright\strut
\textbf{Habitat}\strut
\end{minipage} & \begin{minipage}[t]{0.12\columnwidth}\raggedright\strut
\textbf{Sampling technique}\strut
\end{minipage} & \begin{minipage}[t]{0.12\columnwidth}\raggedright\strut
\textbf{Area sampled}\strut
\end{minipage} & \begin{minipage}[t]{0.12\columnwidth}\raggedright\strut
\textbf{Sampling months}\strut
\end{minipage} & \begin{minipage}[t]{0.12\columnwidth}\raggedright\strut
\textbf{Sampling frequency}\strut
\end{minipage} & \begin{minipage}[t]{0.12\columnwidth}\raggedright\strut
\textbf{Sites sampled\textsuperscript{1}}\strut
\end{minipage} & \begin{minipage}[t]{0.12\columnwidth}\raggedright\strut
\textbf{Sampling efficiency}\strut
\end{minipage}\tabularnewline
\begin{minipage}[t]{0.12\columnwidth}\raggedright\strut
Estuary channels\strut
\end{minipage} & \begin{minipage}[t]{0.12\columnwidth}\raggedright\strut
Fyke trap\strut
\end{minipage} & \begin{minipage}[t]{0.12\columnwidth}\raggedright\strut
250-4700 m\textsuperscript{2}\strut
\end{minipage} & \begin{minipage}[t]{0.12\columnwidth}\raggedright\strut
Feb-July\strut
\end{minipage} & \begin{minipage}[t]{0.12\columnwidth}\raggedright\strut
Bi-weekly\strut
\end{minipage} & \begin{minipage}[t]{0.12\columnwidth}\raggedright\strut
11\strut
\end{minipage} & \begin{minipage}[t]{0.12\columnwidth}\raggedright\strut
20-60\%\strut
\end{minipage}\tabularnewline
\begin{minipage}[t]{0.12\columnwidth}\raggedright\strut
Bay beaches, lagoons\strut
\end{minipage} & \begin{minipage}[t]{0.12\columnwidth}\raggedright\strut
Beach seine\strut
\end{minipage} & \begin{minipage}[t]{0.12\columnwidth}\raggedright\strut
300-700 m\textsuperscript{2}\strut
\end{minipage} & \begin{minipage}[t]{0.12\columnwidth}\raggedright\strut
Feb-Sept\strut
\end{minipage} & \begin{minipage}[t]{0.12\columnwidth}\raggedright\strut
Bi-weekly\strut
\end{minipage} & \begin{minipage}[t]{0.12\columnwidth}\raggedright\strut
32\strut
\end{minipage} & \begin{minipage}[t]{0.12\columnwidth}\raggedright\strut
70-90\%\strut
\end{minipage}\tabularnewline
\begin{minipage}[t]{0.12\columnwidth}\raggedright\strut
Bay subtidal surface waters\strut
\end{minipage} & \begin{minipage}[t]{0.12\columnwidth}\raggedright\strut
Kodiak trawl\strut
\end{minipage} & \begin{minipage}[t]{0.12\columnwidth}\raggedright\strut
8-9000 m\textsuperscript{3}\strut
\end{minipage} & \begin{minipage}[t]{0.12\columnwidth}\raggedright\strut
Apr-Sept\strut
\end{minipage} & \begin{minipage}[t]{0.12\columnwidth}\raggedright\strut
Monthly\strut
\end{minipage} & \begin{minipage}[t]{0.12\columnwidth}\raggedright\strut
12\strut
\end{minipage} & \begin{minipage}[t]{0.12\columnwidth}\raggedright\strut
Unknown but low*\strut
\end{minipage}\tabularnewline
\begin{minipage}[t]{0.12\columnwidth}\raggedright\strut
\strut
\end{minipage} & \begin{minipage}[t]{0.12\columnwidth}\raggedright\strut
\strut
\end{minipage} & \begin{minipage}[t]{0.12\columnwidth}\raggedright\strut
\strut
\end{minipage} & \begin{minipage}[t]{0.12\columnwidth}\raggedright\strut
\strut
\end{minipage} & \begin{minipage}[t]{0.12\columnwidth}\raggedright\strut
\strut
\end{minipage} & \begin{minipage}[t]{0.12\columnwidth}\raggedright\strut
\strut
\end{minipage} & \begin{minipage}[t]{0.12\columnwidth}\raggedright\strut
\strut
\end{minipage}\tabularnewline
\bottomrule
\end{longtable}

*camera-based observations have revealed substantial escapement of fish,
particularly large size classes.

\textsuperscript{1}Includes both index sites (sites repeated each
sampling event) and random sites (sites sampled randomly with
replacement each sampling event).

\begin{itemize}
\item
\item
\item
\item
\item
\item
\item
\item
\item
\item
\item
\item
\item
\item
\item
\item
  \textbf{Figure 1.} Spatial trends in eDNA and visual count data across
  a spatial transect through different marine habitats and sub-habitats
  in Monterey Bay, CA. eDNA counts (expressed as proportions of
  annotated reads) for three replicate samples are plotted for each
  sample site. eDNA counts were significantly~associated with habitat
  for all taxa listed (KW, P\textless{}0.05). Visual counts are~included
  for taxa seen on accompanying dive surveys. Loess curves for visual
  counts are only included for taxa with \textgreater{}15 counts total
  across all surveyed sites, and are not best-fit lines. Plot background
  (white and gray shading) distinguishes the following habitat types,
  moving away from shore: seagrass, kelp forest, shallow sandy bottom,
  rocky reef, deep sandy bottom, and open water. From \emph{17.}
\item
  \emph{}
\item
  \emph{}
\item
  \emph{}
\item
  \emph{}
\item
  \emph{}
\item
  \emph{}
\item
  \emph{}
\end{itemize}

\emph{}\\

\emph{}\\

\emph{}\\

\emph{}\\

\emph{}\\

\emph{}\\

\emph{}\\

\emph{}\\

\textbf{}

\textbf{}\\

\textbf{Figure 2.} Map of the Skagit River (in Northern Puget Sound, WA)
mainstem, estuarine tidal delta, and Skagit Bay, with sites sampled by
different gear types indicated by different shapes.{~ }Squares indicate
Fyke net sites, circles (both open and filled) indicate beach seine
sites, and filled triangles indicate offshore trawl sampling sites. Only
index sites are shown (site which are sampled during every sampling
period).

\textbf{}\\

\textbf{}\\

\textbf{Figure 3:} A schematic illustration of the process of sampling
ecological communities using eDNA and traditional sampling methods.
Boxes correspond to latent states, while arrows and greek letters
represent process contributing to the transitions between states. While
we present only one eDNA path and one traditional sampling path,
recognize that there are many potential variations on the form of this
figure depending on the details of a particular protocol. Note that the
eDNA and traditional sampling branches of this schematic are not
directly comparable (there are no arrows that connect white and black
boxes). From (\emph{16}).

\textbf{}

\textbf{}\\

\textbf{18. Curriculum Vitae:{~}}

{ }See supporting documents

\textbf{19. Letter of Recommendation}: An optional letter of support
from a stock assessment scientist or science director explaining how
your proposal has the potential to improve the accuracy, precision, and
timely of scientific information for the assessment of living marine
resource{~}

\textbf{}

\textbf{}\\

\end{document}
